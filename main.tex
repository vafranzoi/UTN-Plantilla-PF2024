\documentclass[12pt,a4paper,oneside]{report}

	%----------------------------------------------------------
	%	Paquetes utilizados
	%----------------------------------------------------------
%\usepackage[pass,showframe,a4paper,top=30mm,left=30mm,right=20mm,bottom=20mm]{geometry}

\usepackage[a4paper,top=30mm,left=30mm,right=20mm,bottom=20mm,headheight=42pt]{geometry}
\setlength{\headheight}{55pt}
\usepackage[T1]{fontenc}
\usepackage[utf8]{inputenc}
\usepackage{times}
\usepackage[english,spanish]{babel}
\usepackage{amsmath}
\usepackage{graphicx}
\usepackage{subcaption}
\usepackage{lscape}
%\usepackage{showframe}
\usepackage{pdfpages}
\usepackage{apacite}
\usepackage{url}
\usepackage{lipsum}
\usepackage{booktabs}
\usepackage{multirow}
\usepackage[strict]{changepage}
\usepackage{bigstrut}
\usepackage{longtable}
\usepackage{array}
\usepackage{wrapfig}
\usepackage{setspace}
\linespread{1.25} %https://tex.stackexchange.com/questions/65849/confusion-onehalfspacing-vs-spacing-vs-word-vs-the-world
\usepackage{parskip}

\usepackage{fancyhdr} %este y el de abajo son ocupados para hacer los encabezados y pie de pag
\usepackage{etoolbox} % https://tex.stackexchange.com/questions/39988/space-before-chapters-and-contents

\usepackage{titlesec} %para editar chapter y section
\usepackage{titletoc} %para editar tabla de contenidos
%\usepackage{appendixnumbering} %para numerar el anexo
% Definir configuración general de la TOC:

	%----------------------------------------------------------
	%	INICIO DOCUMENTO 
	%----------------------------------------------------------


%Separacion de listados
\let\olditemize\itemize
\def\itemize{\olditemize\itemsep=0pt}

%Separacion de enumerados
\let\oldenumerate\enumerate
\def\enumerate{\oldenumerate\itemsep=0pt}

%Cambio el nombre de las tablas
\def\tablename{Tabla}
\def\listtablename{\'Indice de tablas}%

%Cambio el indentado
\setlength{\parindent}{1.23cm}
%\setlength{\parindent}{0cm}

%Config de los capitulos sin "CAPITULO nro" 
\titleformat{\chapter}{\bfseries\centering}{}{0pt}{\normalsize}
\titlespacing{\chapter}{0cm}{-0.5cm}{0cm}

%Config de la tabla de contenidos
\setcounter{tocdepth}{1} % que aparezcan chapter y section nadamas en TOC

%Da formato a la primera parte de la tabla de contenido, de los titulos que no son capítulo
\titlecontents{chapter}[0pt]{\vskip0pt}{}{}{\titlerule*[.5pc]{.}\thecontentspage}[]
\titleformat{\chapter}{\bfseries\centering}{}{0pt}{\normalsize}

%Config de los secciones
\titleformat{\section}{\bfseries\normalsize}{\thesection}{1cm}{\normalsize}
\titlespacing{\section}{0cm}{0cm}{0cm}


%Config de los subsecciones
\titleformat{\subsection}{\bfseries\normalsize}{\thesubsection}{1cm}{\normalsize}
\titlespacing{\section}{0cm}{0cm}{0cm}


%\pagestyle{fancy}



\pagestyle{fancy}
\fancypagestyle{plain}{}
\renewcommand{\chaptermark}[1]{\markboth{#1}{}}
\renewcommand{\sectionmark}[1]{\markright{#1}{}}

\fancyhead{}

\fancyhead[L]{
	\begin{picture}(0,0)
		\put(40,-30){\includegraphics[height=30pt]{imagenes/LogoUTN_nvgsb.eps}}
\end{picture}\\
\vspace{1.25cm}
{\scriptsize \textbf{Ministerio de Capital Humano\\
Universidad Tecnológica Nacional\\
Facultad Regional Reconquista}\\
\vspace{-0.4cm}}}

\fancyhead[C]{}
\fancyhead[R]{\small{Valentin Aurelio Franzoi\\ Proyecto Final\\}}

\fancyfoot{}
\fancyfoot[L,C]{}
\fancyfoot[R]{\thepage}

\makeatother

\fancypagestyle{toc}{%
	\fancyhf{}%
	\fancyhead[R,C,L]{}%
	\renewcommand{\headrulewidth}{0pt}%
	\renewcommand{\footrulewidth}{0pt}%
	\fancyfoot[R]{\thepage}
}
% Las 3 siguientes lineas no se usan, pero si se usa comando \maketitle son utiles
\title{Diseño eléctrico, programación \\ y control de servicios \\ auxiliares para caldera de \\ central térmica}
\author{Santiago Andrés Franzoi}
\date{Marzo 2021}




\begin{document}
	\begin{titlepage}
		\thispagestyle{empty}
\begin{center}
	
	

		
	{\large{\textsc{{TU NOMBRE EN MAYUSCULA}}}}
	
	\vspace{2cm}
	
	\includegraphics[width=31mm,height=35mm]{imagenes/LogoUTN_nvgsb.eps}
	
	\vspace{2cm}
	
	{\large{\textsc{\textbf{UNIVERSIDAD TECNOLÓGICA NACIONAL}}}}
	
	{\large{Facultad Regional Reconquista}}
	
	\vspace{2cm}
	
	{\Large{\textsc{\textbf{AQUI VA EL TITULO \\ DEL PROYECTO  \\ O TRABAJO PRACTICO\\}}}}
	
	\vfill
	
	{\large{Reconquista, 2025}}
\end{center}
	\end{titlepage}
	\setcounter{page}{2}
	\thispagestyle{toc}

\begin{center}
	\vspace*{2cm}
	\includegraphics[width=31mm,height=35mm]{imagenes/LogoUTN_nvgsb.eps}
\end{center}	
	\vspace{3cm}
	

		\begin{adjustwidth}{3cm}{0cm}
			\hspace*{1.27cm}Proyecto Final presentado en cumplimiento a las exigencias de la\\ Carrera Ingeniería Electromecánica de la Facultad Regional Reconquista\\
		\end{adjustwidth}

	
	

\vspace{7.5cm}

	\begin{tabular}{@{}p{0.25\textwidth}p{0.7\textwidth}@{}}
		Docente Asignatura: & Mg. Ing. Daniel Antón\\
							& Esp. Ing. Gabriel Colman
	\end{tabular}
	
	\begin{tabular}{@{}p{0.12\textwidth}p{0.7\textwidth}@{}}
			Tutores: & Ing. Juan Pablo Suligoy \\
					  & Ing. Mario Ross
	\end{tabular}


\vfill

\begin{center}\begin{large}
		Reconquista, 2025
\end{large}\end{center}

	
	%----------------------------------------------------------

%	\pagestyle{plain}
	\chapter*{\centering{\vspace*{5.7cm}DEDICATORIA}}
	Se lo dedican a toda la gente del repositorio \url{https://github.com/vafranzoi/UTN-Plantilla-PF2024/tree/main}. \\
La plantilla es una modificación de la publicada anteriormente por  \url{https://github.com/SaFRANZOI/proyecto_final_utn_frrq}
\newpage

	
	
	\chapter*{\centering{\vspace*{5.7cm}AGRADECIMIENTOS}}
	\input{hojas-aux/agradecimientos}
	
	\chapter{\centering{RESUMEN-PALABRAS CLAVE}}
	\input{hojas-aux/resumen}
	
	\renewcommand{\contentsname}{ÍNDICE}
	\tableofcontents
	
	\chapter{\centering{LISTA DE ABREVIATURAS}}
	\input{hojas-aux/abreviaturas}

	\chapter{\centering{INTRODUCCIÓN}}
	\input{hojas-aux/introduccion}
	
	\chapter{OBJETIVO GENERAL/OBJETIVOS ESPECÍFICOS}
	\lipsum[0-1]


\section*{Objetivos generales}
\lipsum[0-1]


\section*{Objetivo específico}
\lipsum[0-1]
	
	%Resetear el contador, para que no cuente los titulos que no tienen "capítulo"
\setcounter{chapter}{0}
%Config de los capitulos 
\titleformat{\chapter}{\bfseries\centering}{CAPÍTULO \thechapter. }{0pt}{\normalsize}
\titlespacing{\chapter}{0cm}{-0.5cm}{0cm}



%Para cambiar el "prefijo" (Capitulo / Anexo )
\newcommand{\setupname}[1][\chaptername]{
	\titlecontents{chapter}[0pt]{\vskip0pt}{#1~\thecontentslabel:~}{}{\titlerule*[.5pc]{.}\thecontentspage}[]
	\titleformat{\chapter}{\bfseries\centering}{#1~\thechapter. }{0pt}{\normalsize}
}

\titlecontents{section}[0pt]{\vskip-1pt}{\thecontentslabel~~}{}{\titlerule*[.5pc]{.}\thecontentspage}[]

	\setupname[CAPÍTULO]
	\chapter{CAP1}
	\input{capitulos/capitulo1}
	
	\chapter{CAP2}
	\input{capitulos/capitulo2}
	
	\chapter{CAP3}
	\input{capitulos/capitulo3}
	
	%Da formato a el capítulo de conclusiones, que no lleva "capitulo"
\titlecontents{chapter}[0pt]{\vskip0pt}{}{}{\titlerule*[.5pc]{.}\thecontentspage}[]
\titleformat{\chapter}{\bfseries\centering}{}{0pt}{\normalsize}
%Pone el contador en 0
\setcounter{chapter}{0}
%cambia el formato de los subtitulos
%\titleformat{\section}{\bfseries\normalsize}{\thesection}{1cm}{\normalsize}
%\titlespacing{\section}{0cm}{0cm}{0cm}

	\chapter{CONCLUSIONES}
	\input{hojas-aux/conclusiones}
	
	%----------------------------------------------------------
	%	BIBLIOGRAFIA
	%----------------------------------------------------------
	
	\bibliographystyle{apacite} % estilo de la biblio (probar estilos "ieeetr" o "apacite")
	\renewcommand*{\bibname}{REFERENCIA BIBLIOGRÁFICA}
	\bibliography{./bibliografia} % nombre del archivo .bib
%	\addcontentsline{toc}{chapter}{Bibliografia}
	\newpage
	
	
	\setupname[ANEXO]
	
	 \appendix
	 \renewcommand{\thechapter}{\arabic{chapter}}
	 \setcounter{section}{0}
	 \chapter{PLANOS}
	 \input{capitulos/anexo1}
	 \chapter{CÁLCULOS}
	 \input{capitulos/anexo2}
	 \chapter{VAYAUNOASABER}
	 \input{capitulos/anexo3}





\end{document}
%\pagestyle{fancy}



\pagestyle{fancy}
\fancypagestyle{plain}{}
\renewcommand{\chaptermark}[1]{\markboth{#1}{}}
\renewcommand{\sectionmark}[1]{\markright{#1}{}}

\fancyhead{}

\fancyhead[L]{
	\begin{picture}(0,0)
		\put(40,-30){\includegraphics[height=30pt]{imagenes/LogoUTN_nvgsb.eps}}
\end{picture}\\
\vspace{1.25cm}
{\scriptsize \textbf{Ministerio de Capital Humano\\
Universidad Tecnológica Nacional\\
Facultad Regional Reconquista}\\
\vspace{-0.4cm}}}

\fancyhead[C]{}
\fancyhead[R]{\small{Valentin Aurelio Franzoi\\ Proyecto Final\\}}

\fancyfoot{}
\fancyfoot[L,C]{}
\fancyfoot[R]{\thepage}

\makeatother

\fancypagestyle{toc}{%
	\fancyhf{}%
	\fancyhead[R,C,L]{}%
	\renewcommand{\headrulewidth}{0pt}%
	\renewcommand{\footrulewidth}{0pt}%
	\fancyfoot[R]{\thepage}
}
% Las 3 siguientes lineas no se usan, pero si se usa comando \maketitle son utiles
\title{Diseño eléctrico, programación \\ y control de servicios \\ auxiliares para caldera de \\ central térmica}
\author{Santiago Andrés Franzoi}
\date{Marzo 2021}